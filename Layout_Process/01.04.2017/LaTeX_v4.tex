\documentclass[10pt]{book}
\usepackage[a5paper,top=54pt,bottom=54pt,left=48pt,right=48pt]{geometry}
\usepackage[utf8]{inputenc}
\usepackage{pscyr}
\usepackage[T1,T2A]{fontenc}
\usepackage[english,russian]{babel}
\usepackage{graphicx}
\usepackage{amsmath,amsthm,amssymb}
\usepackage{caption2}
\usepackage{soulutf8,soul}
\usepackage{setspace}
\usepackage{fancybox,fancyhdr}
\pagestyle{fancy}
\fancyhead{}
\fancyhead[LE,RO]{\textbf{\thepage}}
\fancyhead[RE]{\textit{\textsection 39. Формула Тейлора и ряд Тейлора}}
\fancyhead[LO]{\textit{\textsection 39.1. Формула Тейлора для функций многих переменных}}
\fancyfoot{}
\renewcommand{\headrulewidth}{0pt}
\usepackage{amssymb}
\renewcommand{\geq}{\geqslant}
\renewcommand{\leq}{\leqslant}
\setcounter{page}{8}
\usepackage{pgfpages}
\pgfpagesuselayout{2 on 1} [a4paper,landscape,border shrink=5pt]

%font
\fontfamily{lh}
\selectfont


\begin{document}

	\begin{spacing}{0.85}
	   \noindent ренцирования сложной функции, если их аргументы, как и в (39.3), \linebreak
	   линейно зависят от \textit {t}. Поэтому приведенное выше доказательство \linebreak
	   формулы Тейлора полностью сохраняется и для этого случая.
	   
	   Формулу Тейлора (39.1) можно доказать и при еще более слабых \linebreak
	   ограничениях, однако это потребовало бы более тонкого доказатель-\linebreak
	   ства, и мы не будем на этом останавливаться (для случая одной пере-\linebreak
	   менной см. упражнение 1 в \S 13).
	 
	   Формулу (39.1) можно несколько обобщить и в другом смысле: \linebreak
	   не требовать, чтобы функция \textit {f} была определена во всех точках не- \linebreak
	   которой окрестности точки ($x_0$, $y_0$), а рассматривать эту формулу \linebreak
	   лишь при фиксированных $\Delta x$ и $\Delta y$. Именно если функция \textit {f} определена и имеет дифференцируемые частные производные до порядка \linebreak
	   \textit {m} \textemdash~1 включительно в каждой точке отрезка с концами в точках \linebreak
	   ($x_0$, $y_0$) и ($x_0$ +  $\Delta x$, $y_0$ + $\Delta y$), то формула (39.1) также остается спра-
	    \linebreak ведливой вместе с доказательством.
	   
	   Из всего сказанного следует, что если функция \textit {f} определена в выпуклой области \textit {G} (см. п. 18.2) и имеет \textit{G} дифференцируемые частные производные порядка \textit {m} \textemdash~1, то для любых двух точек ($x_0$, $y_0$) $\in$ \textit {G} \linebreak
	   и ($x_0$ +  $\Delta x$, $y_0$ + $\Delta y$) $\in$ \textit{G} имеет место формула Тейлора (39.1).
	   
	   Для справедливости же формулы Тейлора (39.9), кроме диффе- \linebreak
	   ренцируемости производных порядка \textit {m} \textemdash~1 в окрестности точки \linebreak
	   ($x_0$, $y_0$), достаточно лишь потребовать, чтобы производные порядка \textit {m}
	   были непрерывны только в точке ($x_0$, $y_0$).
	   
	   Мы не стали всего этого сразу оговаривать для простоты форму-\linebreak
	   лировок и доказательств теоремы 1 и ее следствия.
		
		Подчеркнем еще, что в формуле (39.9) $r_m$($\Delta x$, $\Delta y$) = $o(p^m)$ \linebreak
		не в смысле предела по любому фиксированному направлению, как мо- \linebreak
		жет показаться на первый взгляд из приведенного доказательства,\linebreak
		а в более сильном смысле, в смысле предела в точке  ($x_0$, $y_0$) (почему?).
	\end{spacing}		
		\setlength{\parskip}{0.05cm}
	\begin{spacing}{0.9}
		\small{\so{Упражнение 1.} Пусть функция \textit {f}$(x, y)$ определена и непрерывна \linebreak
		вместе со своими частными производными до порядка \textit {m}  включительно в не- \linebreak
		которой окрестности точки ($x_0$, $y_0$). Доказать, что ее \textit {многочлен Тейлора} по- \linebreak рядка \textit {m}, т. е. многочлен \small{
		$$\textit{P(x, y)} = \sum\limits_{k = 0}^m \frac{1}{k!} \bigl[(x- x_0) \frac{\partial}{\partial x} + (y- y_0) \frac{\partial}{\partial y}\bigr]^{\{k\}} \textit {f}(x_0, y_0) ,$$}
	
		\noindent является многочленом наилучшего приближения функции $\textit{f}(x, y)$ <<в беско- \linebreak
		нечно малой окрестности точки ($x_0$, $y_0$)>>. Это означает следующее: каков бы ни \linebreak
		был многочлен $Q(x, y)$ степени не больше \textit {m} (в каждом его члене сумма показа- \linebreak
		телей степени y переменных x и y должна не превышать числа \textit{m}) такой, что
		 \small{	$$\textit {f(x, y)} = \textit{Q(x, y) + $o(p^n)$,\quad n $\geq$ m,}$$}
		\noindent где  \small{ $$p=\sqrt{(x - x_0)^2 + (y-y_0)^2,}$$ } он совпадает с указанным многочленом Тейлора \textit{P(x, y)} функции  \textit{f(x,y)}. 
		}
	\end{spacing}	
	 %---------------------------------------------------------------------------------------------------------
	    Все сказанное переносится и на случай функций любого чис- \linebreak
	    ла переменных.
	   
	    \textit{\textbf{Теорема 1$'$.} Если функция n переменных $y = \textit{f}(x_1, ..., x_n)$ \linebreak определена и непрерывна вместе со всеми своими частными производ- \linebreak
	    ными до порядка \textit {m} включительно в некоторой окрестности точки
	    	$x^{(0)}$ = $(x_{i}^{(0)})^{*)}$, то справедлива формула}
	   
	\begin{spacing}{0.5}
	   
	    	$$\Delta y = \textit{f}(x_{1}^{(0)} + \Delta x_{1}, ..., x_{n}^{(0)}+\Delta x_{n}) - \textit{f}(x_{1}^{(0)}, ...,  x_{n}^{(0)}) = $$
	    	$$ = \sum\limits_{k = 1}^{m-1} \frac{1}{k!} \bigl( \frac{\partial}{\partial x_1} \Delta x_1 + ... +\frac{\partial}{\partial x_n} \Delta x_n \bigr)^{\{k\}} \textit{f}(x^{(0)}) + r_{m-1} (\Delta x), \eqno(39.18) $$ 
	
	    \noindent \textit {где}
	    	$$r_{m-1}(\Delta x) =$$	
	    	$$=\frac{1}{m!} \bigl( \frac{\partial}{\partial x_1} \Delta x_1 + ... + \frac{\partial}{\partial x_n} \Delta x_n \bigr)^{\{m\}} \textit{f}(x_{1}^{(0)} + \theta \Delta x_1, ...,x_{n}^{(0)} + \theta \Delta x_n ),$$
	  
	    	$$0 < \theta < 1, \quad \Delta x = (\Delta x_1, ...,\Delta x_n), 	\eqno (39.19)$$
	    %\end{spacing}
	    
	    \noindent \textit {а также формула}
	    	$$\Delta y  = \sum\limits_{k = 1}^{m} \frac{1}{k!} \bigl( \frac{\partial}{\partial x_1} \Delta x_1 + ... +\frac{\partial}{\partial x_n} \Delta x_n \bigr)^{\{k\}} \textit{f}(x^{(0)}) + r_{m} (\Delta x), \eqno (39.20)$$
	   
	    \noindent \textit {где $r_{m} (\Delta x) $ можно записать в каждом из следующих видов:}
	    
	    \textit {либо}
	    	$$r_{m}(\Delta x) = \sum\limits_{m_1+ ... +m_n = m} \varepsilon_{m_1 ... m_n}(\Delta x) 
	    	\Delta x_1^{m_1} ... \Delta x_n^{m_n}, \eqno (39.21)$$
	     	
	    \noindent \textit {где}
	    $$\lim_{\rho \to 0} \varepsilon_{m_1, ..., m_n}(\Delta x) = 0, \quad
	    \rho = \sqrt{\sum\limits_{i=1}^n \Delta x_i^{2}},$$
	    
	    \textit {либо} 
	    
	    $$r_{m} (\Delta x) = \varepsilon (\Delta x) \rho^m,\quad \lim_{\rho \to 0} \varepsilon(\Delta x) = 0, \eqno (39.22) $$
	    
	    \noindent \textit {т.е.} 	
	    	$$r_{m} (\Delta x) = o(\rho^m),\quad \rho \to 0.$$
	    	
	    \textit {Наконец, через дифференциалы формулу (39.20) можно за- \linebreak писать в виде}
	    	$$\Delta y = \sum\limits_{k = 1}^{m} \frac{1}{k!} d^k \textit{f}(x^{(0)}) + r_{m} (\Delta x). \eqno (39.23)$$	
	\end{spacing}
		\vfill {
			\begin{spacing}{0.4}	
	   			\noindent\line(1,0){45}
	    		\linebreak

	    		\footnotesize{*) Эти ограничения можно несколько ослабить аналогично тому, как это было указано выше в случае функций двух переменных.}
			\end {spacing}	
		}
\end{document}
