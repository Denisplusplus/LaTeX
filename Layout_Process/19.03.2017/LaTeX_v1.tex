\documentclass{article}
\usepackage[utf8]{inputenc}
\usepackage[english, russian]{babel}
\usepackage{pscyr}
\usepackage[T1, T2A]{fontenc}
\newcommand\mes{\mathop{\mathrm{mes}}\nolimits}
%\usepackage[landscape]{geometry}
\begin{document}
	\noindentреценирования сложной функции, если их аргументы, как и в (39.3),
	линейно зависятот \textit {t}. Поэтому приведенное выше доказательство
	формулы Тейлора полностью сохраняется и для этого случая.
	 
	Формулу Тейлора (39.1) можно доказать и при еще более слабых
	ограничениях, однако это потребовало бы более тонкого доказательства,
	 и мы не будем на этом останавливаться (для случая одной пере-
	менной см. упражнение 1 в \S 13).
	 
	Формулу (39.1) можно несколько обобщить и в другом смысле:
	не требовать, чтобы функция \textit {f} была определена во всех точках не
	которой окрестности точки ($x_0$, $y_0$), а рассматривать эту формулу
	лишь при фиксированных $\Delta x$ и $\Delta y$. Именно если функция \textit {f} определена и имеет дифференцируемые частные производные до порядка 
	\textit {m} \textemdash~1 включительно в каждой точке отрезка с концами в точках 
	($x_0$, $y_0$) и ($x_0$ +  $\Delta x$, $y_0$ + $\Delta y$), то формула (39.1) также остается спра-
	ведливой вместе с доказательством.
	 
	Из всего сказанного следует, что если функция f определена в вы-
	пуклой области G (см. п. 18.2) и имеет G дифференцируемые част-
	ные производные порядка \textit {m} \textemdash~1, то для любых двух точек ($x_0$, $y_0$) $\in$ G 
	и ($x_0$ +  $\Delta x$, $y_0$ + $\Delta y$) $\in$ G имеет место формула Тейлора (39.1)
	 
	Для справедливости же формулы Тейлора (39.9), кроме дифференцируемости производных порядка \textit {m} \textemdash~1 в окрестности точки 
	($x_0$, $y_0$), достаточно лишь потребовать, чтобы производные порядка m
	были непрерывны только в точке ($x_0$, $y_0$).
	 
	Мы не стали всего этого сразу оговаривать для простоты форму
	лировок и доказательств теоремы 1 и ее следствия.
	 
	Подчеркнем еще, что в формуле (39.9) $r_m$($\Delta x$, $\Delta y$) = $o(p^m)$
	не в смысле предела по любому фиксированному направлению, как может показаться
	на первый взгляд из приведенного доказательства, а вболее сильном смысле, в смысле предела
	в точке  ($x_0$, $y_0$) (почему?).

	Упражнение 1. Пусть функция $f(x, y)$ определена и непрерывна вместе со свомими частными проивзодными
	до порядка \textit {m}  включительно в некоторой окрестности точки ($x_0$, $y_0$). Доказать, что 
	ее многочлен Тейлора порядка m, т. е. многочлен


	\textit{P}(x, y) = $\sum\limits_{k = 0}^m \frac{1}{k!} \bigl[(x- x_0) \frac{\partial}{\partial x} +
	 (y- y_0) \frac{\partial}{\partial y}\bigr]^{\{k\}} \textit {f}(x_0, y_0)$,


	является многочленом наилучшего приближения функции $f(x, y)$ "в бесконечно малой
	окрестности точки ($x_0$, $y_0$)". Это означает следующее: каков бы ни был многочлен $Q(x, y)$ степени
	не больше m (в каждом его члене сумма показателей степени y переменных x и y должна не превышать числа m) такой, что

	\textit {f}(x, y) = \textit{Q}(x, y) + $o(p^m)$, \textit{n $\geq$ m,}


	где
		
		$p=\sqrt{(x - x_0)^2 + (y-y_0)^2,}$

	он совпадает с узказанным многолченом Тейлора $P(x, y)$ функции $f(x,y)$.
 
\end{document}
