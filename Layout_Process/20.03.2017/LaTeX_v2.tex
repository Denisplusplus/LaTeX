\documentclass[10 pt]{article}
\usepackage[utf8]{inputenc}
\usepackage[english, russian]{babel}
\usepackage{pscyr}
\usepackage[T1, T2A]{fontenc}
\newcommand\mes{\mathop{\mathrm{mes}}\nolimits}
%\usepackage[landscape]{geometry}
\begin{document}
	Все сказанное переносится и на случай функций любого числа переменных.
	
	Теорема 1'. Если функция n переменных $y = f(x_1, ..., x_n)$ определена и непрерывна вместе
	со всеми своими частными производными до порядка m включительно в некоторой окрестности точки
	$x^{(0)}$ = $(x_{i}^{(0)})^{*)}$, то справедлива формула \\
	$\Delta y = f(x_{1}^{(0)} + \Delta x_{1}, ..., x_{n}^{(0)}+\Delta x_{n}) - f(x_{1}^{(0)}, ...,  x_{n}^{(0)}) = \\ = \sum\limits_{k = 1}^{m-1} \frac{1}{k!} \bigl( \frac{\partial}{\partial x_1} \Delta x_1 + \frac{\partial}{\partial x_n} \Delta x_n \bigr)^{\{k\}} f(x^{(0)}) + r_{m-1} (\Delta x) $, (39.18)
 
\end{document}
